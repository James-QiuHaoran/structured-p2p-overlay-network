% TEMPLATE for Usenix papers, specifically to meet requirements of
%  USENIX '05
% originally a template for producing IEEE-format articles using LaTeX.
%   written by Matthew Ward, CS Department, Worcester Polytechnic Institute.
% adapted by David Beazley for his excellent SWIG paper in Proceedings,
%   Tcl 96
% turned into a smartass generic template by De Clarke, with thanks to
%   both the above pioneers
% use at your own risk.  Complaints to /dev/null.
% make it two column with no page numbering, default is 10 point

% Munged by Fred Douglis <douglis@research.att.com> 10/97 to separate
% the .sty file from the LaTeX source template, so that people can
% more easily include the .sty file into an existing document.  Also
% changed to more closely follow the style guidelines as represented
% by the Word sample file. 

% Note that since 2010, USENIX does not require endnotes. If you want
% foot of page notes, don't include the endnotes package in the 
% usepackage command, below.

% This version uses the latex2e styles, not the very ancient 2.09 stuff.

% Updated July 2018: Text block size changed from 6.5" to 7"

\documentclass[letterpaper,twocolumn,10pt]{article}
\usepackage{usenix2019, epsfig, endnotes, amsmath, cleveref, enumitem, algorithm}
\usepackage[noend]{algpseudocode}

\makeatletter
\def\BState{\State\hskip-\ALG@thistlm}
\makeatother
\algnewcommand\algorithmicforeach{\textbf{for each}}
\algdef{S}[FOR]{ForEach}[1]{\algorithmicforeach\ #1\ \algorithmicdo}
\newcommand{\var}{\texttt}

\crefformat{section}{\S#2#1#3} % see manual of cleveref, section 8.2.1
\crefformat{subsection}{\S#2#1#3}
\crefformat{subsubsection}{\S#2#1#3}

\begin{document}

%don't want date printed
\date{}

%make title bold and 14 pt font (Latex default is non-bold, 16 pt)
\title{\Large \bf HGFRR: Hidden Geographical Fractal Random Ring}

%for single author (just remove % characters)
%\author{
%{\rm Your N.\ Here}\\
%Your Institution
%\and
%{\rm Second Name}\\
%Second Institution
% copy the following lines to add more authors
% \and
% {\rm Name}\\
%Name Institution
%} % end author

\author{
{\rm Anonymous Authors}\\
}

\maketitle

% Use the following at camera-ready time to suppress page numbers.
% Comment it out when you first submit the paper for review.
\thispagestyle{empty}

\subsection*{Abstract}
Blockchain systems are used to record transactions among parties without a central authority. Since the very first instance of a blockchain systems, Bitcoin, tremendous amount of blockchain systems with various consensus protocols are designed and implemented to achieve fast transaction rate. With the transaction rate increasing, experiments and studies show that network layer become the bottleneck on the way to further improve blockchain system efficiency. However, current blockchain systems are based on unstructured, structured, or hybrid Peer-to-Peer networks. Gossiping messages on such networks creates repeated messages and leads to traffic congestion when transaction rates grows. In addition, lack of attention paid to geographical locality and security in the network layer also limits the improvement of the P2P network underlying blockchain systems.

In this paper, we present \textbf{H}idden \textbf{G}eographical \textbf{F}ractal \textbf{R}andom \textbf{R}ing (\xxx). It constructs and maintains a recursive DHT-ring like structure according to geographical proximity of nodes. The broadcast operation on such a network which contains $N$ nodes achieves $O(N)$ in terms of message complexity, and $O(logN)$ in terms of time complexity. Security issues are also addressed to protect the anonymity of nodes. Evaluation shows that \xxx outperforms typical P2P network in blockchain systems in both time complexity and messages complexity. Consequently, the throughput of the blockchain system can be further improved around 1.4X to 2.1X. Source code of the implementation of \xxx (in C++) will be available on GitHub once appropriate.

\section{Introduction}
% Flow:
%
% start from talking about current blockchain systems, transaction rate is getting increasing
% ->
% problem is: current blockchain systems are based on kademlia (dht) structure, which is efficient on node discovery and data look up, however, broadcast suffers traffic congesion and inefficient convergence (gossip)
% ->
% many improvements are made on gossip, not quite efficient, still suffers bandwidth problem
% ->
% less p2p network are designed for blockchain, talk about p2p networks in other domains
% ->
% analysis blockchain needs and those p2p networks does not fit\\
% ->
% moreover, current blockchain systems didn't address geographical feature
% ->
% moreover, current blockchain systems didn't address security problem from the p2p layer
% ->
% talk about hgfrr, give a overview, analysis result, implementation overview, evaluation result
% ->
% summarize our feature \& contributions
% ->
% paper structure overview

A blockchain is essentially a distributed ledger that permanently records the transactions among parties \cite{iansiti2018truth}. The transactions recorded are verifiable and resistant to modification. This feature of security has led to the emergence of cryptocurrencies which leverages the blockchain system as their cornerstone, the most salient example being Bitcoin \cite{nakamoto2008bitcoin}. Despite the popularity of cryptocurrencies, general and all-purpose blockchain systems that accommodate various applications, such as Ethereum \cite{wood2014ethereum} have been proposed. However, although Ethereum is a Turing-complete system \cite{wood2014ethereum}, it is in essence designed for the cryptocurrency based on it. Hence the Proof-of-Work (PoW) consensus has been utilized used to support the valuation of the cryptocurrency in the socioeconomic sense, which results in poor efficiency. To facilitate general-purpose applications in a more efficient manner, other consensus protocols have been proposed to improve the performance of the blockchain system in different scenarios such as Proof-of-Stake \cite{kiayias2017ouroboros} and Proof-of-Luck \cite{milutinovic2016proof}. Hyperledger Sawtooth \cite{sawtooth} utilizes Intel Software Guard eXtentions (SGX, introduced below) to trust nodes and proposes Proof-of-Elapsed Time believed to be highly efficient, yet scalibility might not be a critical performance in this consensus. GEEC is a new public blockchain protocol which leverages the strong confidentiality and integrity of the  hardware \cite{2018arXiv180802252C}. Its throughput is comparable to Visa's and is 1.7X to 88.5X higher than Ethereum \cite{buterin2014next}, EOS \cite{eosio}, and Intel-PoET \cite{prisco2016intel}.

With the increasing of the blockchain system efficiency and transaction rate, broadcast frequency is growing. However, current blockchain systems are based on Distributed Hash Table (DHT) structure. It is efficient in terms of node discovery and data look up, however, broadcasting messages on top of this kind of network structure suffers from traffic congestion and inefficient convergence problems when transaction rate gets higher. Our key insights is that the Gossip algorithm used to broadcast a message does not fit in the demand for P2P networks from blockchain systems. Although many improvements has been made on Gossip algorithm such as adding unique message ID, using Time-to-Live field to control flooding, using pull-version sending mechanism to reduce repeated messages, our evaluations show that they are not efficient, and still suffer from traffic congestion problem. 

Blockchain systems' requirements are different from other Peer-to-Peer applications: (i) on-demand streaming allows users to look up data in the P2P network and download stream data from the source, e.g. BitTorrent-based streaming systems like BASS \cite{dana2005bass}, Peer-Assisted \cite{carlsson2007peer}, LiveBT \cite{lv2007livebt}, and Give-To-Get \cite{mol2008give}; (ii) audio/video conferencing applications deal with small scale point-to-point connected networks which requires low latency, e.g. Skype \cite{baset2004analysis}; (iii) peer-to-peer file sharing makes efficient indexing and searching possible, e.g. Napster \cite{saroiu2003measuring}, Gnutella \cite{ripeanu2001peer}, and KaZaA \cite{good2003usability}; (iv) video streaming applications enables single-source broadcasting efficient, e.g. SplitStream \cite{castro2003splitstream}, Bullet \cite{kostic2003bullet}, and ChainSaw \cite{pai2005chainsaw}. (i) and (ii) are not relevant to the context of blockchain systems since nodes in a blockchain system network should be in a large scale and broadcasting a message is an active operation instead of searching and downloading data. (iii) and (iv) are more similar to blockchain systems' use case. However, P2P file sharing is not real-time and the broadcast model in a blockchain system is not indexing and searching. In video streaming, time is stringent and the network size can be large-scale. However, it is a data or bandwidth-intensive communication which means control messages in a broadcast operation are relatively small compared to the data to transmit.

Through our study, we summarized two requirements for the underlying P2P network from blockchain systems: peer discovery and message dissemination. There are related works focused on improving peer discovery and message dissemination in a separated way. For example, DHT-based protocols such as Kademlia, Chord, Pastry, Tapestry, CAN focus on peer discovery, while Gossip-based protocols and tree-based protocols focus on message dissemination. There is very few works addressing both issues at the same time. To fill to gap, our key insight is that broadcasting using the DHT-based structure can improve the broadcast efficiency in terms of both time complexity and message complexity.

We implemented our idea in HGFRR, which is a \textbf{H}idden \textbf{G}eographical \textbf{F}ractal \textbf{R}andom \textbf{R}ing structured P2P network. HGFRR contains multi-level fractal random rings. Each ring at the lower level will have a couple of contact nodes which is randomly selected to represent the ring in the upper level ring. Unlike DHT-based protocols which indexes the whole network as a ring, HGFRR recursively constructs rings based on the proximity of peers and the number of peers in a ring. The broadcast on HGFRR is recursively performed on each ring. The in-ring broadcast uses the k-ary distributed spanning tree formed within the ring. Both the proof and evaluation show that the broadcast operation in HGFRR is more efficient than the P2P network in Ethereum, in terms of time complexity and message complexity. The message complexity of our network with $N$ nodes is $O(N)$ and time complexity of message broadcast is logarithm, which are both better than extant work.

The paper makes the following contributions. First, the requirements for the P2P networks from blockchain systems are summarized. Second, a new network protocol that addresses both peer discovery and message dissemination. Third, HGFRR is the first P2P network in blockchain systems that addresses geographical locality and security problems. Fourth, HGFRR is implemented in C++, which is portable and cross-platform.

The remaining of this paper is structured as follows: the background, the design overview, proof of analysis, implementation, and evaluation, related works. [TODO]

\section{Background and Motivation}
This is the background and motivation.

\subsection{Peer-to-Peer Overlay Networks}

Talk about the overview on various peer-to-peer networks. Structured vs unstructured.

\subsubsection{Node Discovery}

Talk about the node discovery in p2p networks. DHT. Chord, CAN, Tapestry, Pastry.

\subsubsection{Broadcast}

Talk about the broadcast in p2p networks. Various version of Gossip. Broadcast in tree-based network.

\subsection{P2P Networks in Blockchain Systems}

Talk about current p2p networks in a blockchain system, blockchain system's demand.

\subsection{Trusted Execution Environment (TEE)}

Talk about trusted execution environment, Intel SGX.

\subsection{Design Motivation}

Talk about the design motivation, kind of combining the two.

\section{HGFRR Design}
In this section, we first introduce the design principles of \xxx (Section \cref{principles}). Then we present the topology of the network (Section \cref{topo}), how the structure is formed (Section \cref{formation}) and maintained (Section \cref{maintain}), and the broadcast algorithm (Section \cref{broadcast}). Security issues are also addressed in the design of \xxx (Section \cref{security}).

\subsection{Design Principles} \label{principles}

The design of \xxx as a Peer-to-Peer network layer under a blockchain system follows four principles:
\noindent
\begin{itemize}[noitemsep, topsep=0pt]
	\item Fair: An unfair P2P network may ascend free-riders, frustrate majority of users and consequently lead to instability of the network \cite{naghizadeh2016improving}.
	\item Self-organizing: No central server should be responsible for organizing the structure of the network. In other words, the decentralization nature of the P2P network should not be affected.
	\item Anonymous: Each node in the network and the network topology should be hidden from the outsider so that target attack can be avoided to some extent.
	\item Robust in the dynamic environment: The network is unstable due to the frequent disconnection or node-join. The structure of the network should be easy to maintain in a distributed manner.
\end{itemize}

%\item Efficient on broadcast: The converge time of each message to broadcast should be as small as possible. Therefore, the number of concurrent messages on flight will be reduced.
%\item Scalable to large number of nodes: The P2P network should be able to scale up to tremendous number of nodes as a public block chain may grow up to millions of nodes.

\subsection{Topology} \label{topo}

Before presenting the protocol, several key concepts should be defined clearly:
\begin{itemize}[noitemsep, topsep=0pt]
	\item Node: One instance of a server/virtual machine in the network;
	\item Ring: A group of nodes connected in a ring-like structure.
	\item Contact Node: the node on the ring who is in charge of adding new nodes, contacting the nodes in the upper level of the network, and broadcasting the message.\\
\end{itemize}

\begin{figure}[t]
	\includegraphics[width=0.47\textwidth]{figures/topo.jpg}
	\caption{Topology Illustration of a 3-level \xxx Network}
	\label{fig:topo}
\end{figure}

The network topology of \xxx is basically a fractal-ring structure, where lower level rings reside on higher level rings (See Figure \ref{fig:topo}) in a recursive way. At the top level resides the largest ring where several sub-ring resides on while at the bottom level resides the smallest rings. The figure shows a network of 3 levels. Level 2 contains the largest ring. On the largest ring, there are three sub-rings of 2 levels. Level 1 contains the second largest rings and level 0 contains the smallest rings. The nodes in red are contact nodes of each ring. They are elected to be normal nodes of the upper level ring.

\subsubsection{Structure Formation} \label{formation}

When a new node wants to join the network, it will first send ping-messages to each of the contact nodes of the largest ring (at the top level). The new node will pick the contact node with the shorted response time to send a join-message. The contact node will then decide which sub-ring it should add this new node to, by sending the node information of contact nodes of each sub-ring back to the new node. Recursively, the new node will then choose the sub-ring which is optimal in terms of response time. And the contact node of the sub-ring will then introduce the new node to the sub-sub-ring. In the end, the contact node of a ring at the bottom level will then add the new node to the ring. If the number of node on the ring that the new node join to exceeds the threshold, then this ring will transform to a two-level ring (See Figure TODO), i.e. several groups of nodes on the large ring will form several rings. The transformation will be further elaborated in Section \cref{maintain}. After a new node joins the network, the contact node of this ring will broadcast within ring this node's information. The member nodes of this ring will then tell their information by sending welcome-messages to the new node.

The upper level rings are formed by contact node election. When a ring is first formed, the first member node of the ring will be the contact node of this ring. Each generation of contact nodes have their term of service. At the end of the term, one of the contact nodes will generate random IDs from all member node IDs. This election result will be dispersed within ring, and multi-casted to the contact nodes of the upper level ring and the lower level rings.

\begin{figure}[t]
	\includegraphics[width=0.47\textwidth]{figures/topo-vertical.jpg}
	\caption{Vertical Illustration of \xxx Network Structure. The figure illustrated the relationship between two levels next to each other.}
	\label{fig:topo}
\end{figure}

\begin{algorithm}
	\caption{Bootstrap}\label{euclid}
	\begin{algorithmic}[1]
		\Procedure{Node Join}{}
		\State $\var{response} \gets \var{queryDNSSeeds}()$
		\State $\var{contactNodeList} \gets \var{response.nodes}$
		\State $\var{level} \gets \var{response.topLevel}$
		\BState \emph{loop}:
		\If {$\var{level} = \var{0}$}
		\State $\textbf{break}$
		\Else
		\State $\var{optimalMetric} \gets \var{0}$
		\For {$\var{contactNode} : \var{contactNodeList}$}
		\State $\var{metric} \gets \var{testProximity}(\var{contactNode})$
		\If {$\var{metric} > \var{optimalMetric}$}
		\State $\var{optimalMetric} \gets \var{metric}$
		\State $\var{targetNode} \gets \var{contactNode}$
		\EndIf
		\EndFor
		\State $\var{response} \gets \var{queryNode}(\var{contactNode}, \var{level})$
		\State $\var{contactNodeList} \gets \var{response.nodes}$
		\State $\var{level} \gets \var{level}\var{-1}$
		\State \textbf{goto} \emph{loop}.
		\EndIf
		\State $/* \text{contact node of level 0 ring broadcast node-join}$
		\State $\text{\; * message withing the ring */}$
		\State $\var{this.}\var{nodeTable.insert}(\var{recvMsg.node})$
		\If {$\var{recvMsg.node.contactNode} = \var{true}$}
		\State $\var{contactNode} \gets \var{recvMsg.node}$
		\State $\var{contactNodeList.insert}(\var{contactNode})$
		\EndIf
		\EndProcedure
	\end{algorithmic}
\end{algorithm}

\subsubsection{Structure Maintenance} \label{maintain}

Each node on the bottom ring will send heart-beat messages to its successor and predecessor to check the aliveness of them. Once a node are not responding to the heart-beat message after the timeout value, the node will double check the liveness of this node with its, e.g. if A's predecessor B does not respond, A will check with the predecessor of B. If they agree that this node left the network (intentionally or accidentally), the information will be disseminated to the ring and this node will be officially removed from the network. If the missing node is the contact node, then the next generation of contact nodes will be elected. If the number of nodes on the ring is smaller than the lower limit, a transformation from right to left in Figure [TODO] will be performed.

\iffalse
\begin{algorithm}
	\caption{Maintenance}\label{euclid}
	\begin{algorithmic}[1]
		\State {\text{// This function will be called every \var{TIME\_INTERVAL}}}
		\Function{detect\_node\_left}{}
		\If{$\var{liveness\_check\_predecessor}() = \var{false}$}
		\State $\var{msg} \gets \var{constructNodeLeaveMSG}()$
		\State $\var{this.inRingBroadcast(msg)}$
		\EndIf
		\If{$\var{liveness\_check\_successor}() = \var{false}$}
		\State $\var{msg} \gets \var{constructNodeLeaveMSG}()$
		\State $\var{this.inRingBroadcast(msg)}$
		\EndIf
		\EndFunction
		\\
		\Function{liveness\_check\_predecessor}{}
		\State $\var{ret} = \var{sendHeartBeatMSG(this.predecessor)}$
		\If{$\var{ret.type} = \var{TIMEOUT}$}
		\State $\var{status} \gets \var{confirmWithPrePredecessor}()$
		\State \Return $\var{status}$
		\EndIf
		\EndFunction
		\\
		\Function{liveness\_check\_successor}{}
		\State $\var{ret} = \var{sendHeartBeatMSG(this.successor)}$
		\If{$\var{ret.type} = \var{TIMEOUT}$}
		\State $\var{status} \gets \var{confirmWithSucSuccessor}()$
		\State \Return $\var{status}$
		\EndIf
		\EndFunction
	\end{algorithmic}
\end{algorithm}
\fi

\subsection{Broadcast} \label{broadcast}

\textbf{Broadcast} is the process of disseminating a message from any node to the whole network. When a node wants to send a message to the whole network, it will first send broadcast-message to one of the contact nodes of the ring it resides on. Then the message will be routed to two directions: one direction is downwards, i.e. the contact node will broadcast the message in the ring and recursively in the sub-rings; the other direction is upwards, i.e. the contact node will send broadcast-message to one of the contact nodes of the upper level ring. Recursively, the broadcast-message will be received by one of the contact nodes of the largest ring. Then the contact node will broadcast recursively in the sub-rings. Till the bottom level, each node in the fractal ring will receive this message.

The k-ary distributed spanning tree method \cite{el2003efficient} is used to broadcast message in a ring. Details will be presented in the subsection. Based on this method, the time complexity of a broadcast operation will be $O(logN)$ and message complexity will be $(O(N))$, which are currently the best among related works.

\subsubsection{Broadcast Within-Ring Mechanism}

In-ring broadcast is based on the k-ary distributed spanning tree method. The basic idea is that the broadcast starter will first generate a random number $k$, and then a k-ary spanning tree can be formed in a distributed manner. Broadcast will then be triggered from the root to every node in the tree. The reason we choose to randomize the parameter k is that the network should be hidden from the attacker. If it keeps using the same parameter k, the routing pattern will be known easily by watching the network activities for a long time. The spanning tree are formed by using the broadcaster (which is numbered 0) as the root. Node 0 will connect to node $0+k^0$, $0+k^1$, $0+k^2$, and so on. Similarly, node 1 will connect to $1+k^0$, $1+k^1$, $1+k^2$, and so on. The pattern is: node $i$ will connect to $i+k^0$, $i+k^1$, $i+k^2$, and so on. The overall time complexity of this method will be $O(logN)$, where $N$ is the number of nodes in the ring.

\begin{lstlisting}
// broadcast data from any node in the network
void broadcast(data) {
  level = 0;
  contactNode = selectFromContactNodes(level);
  msg = wrapMsg(data);
  sendTo(contactNode, msg);
  return;
}

// broadcast upwards
void broadcast_up(currLvl, data) {
  level = currLvl + 1;
  contactNode = selectFromContactNodes(level);
  msg = wrapMsg(data);
  sendTo(contactNode, msg);
  return;
}

// broadcast within ring
void broadcast_down(currLvl, msg) {
  endID = this.getNodeTableSize(currLvl);
  i = 0;
  nodeOrder = msg.nodeOrder;
  k = msg.k;
  while (nodeOrder + pow(k, i) < endID) {
    i++;
    if (pow(k, i) <= nodeOrder) {
      continue;
    } else {
      targetID = nodeOrder + pow(k, i);
      if (targetID > endID)
        targetID -= endID + 1;
      receiver = this.getPeer(targetID, currLvl);
      msg = wrapMsg(msg);
      sendTo(receiver, msg);
    }
  }	
  return;
}
\end{lstlisting}

\iffalse
\begin{algorithm}[t]
	\caption{Broadcast}\label{broadcast}
	\begin{algorithmic}[1]
		\Function{broadcast}{\var{data}}
		\State $\var{level} \gets \var{0}$
		\State $\var{contactNode} \gets \var{selectFromContactNodes(level)}$
		\State $\var{message} \gets \var{wrapMessage(data)}$
		\State $\var{sendTo(contactNode, message)}$
		\EndFunction
		\\
		\Function{broadcast\_up}{\var{currentLevel, data}}
		\State $\var{level} \gets \var{currentLevel+1}$
		\State $\var{contactNode} \gets \var{selectFromContactNodes(level)}$
		\State $\var{message} \gets \var{wrapMessage(data)}$
		\State $\var{sendTo(contactNode, message)}$
		\EndFunction
		\\
		\Function{broadcast\_within\_ring}{\var{currentLevel, message, sentIDs, k}}
		\State $\var{endID} \gets \var{this.getNodeTableSize(currentLevel)}$
		\State $\var{i} \gets \var{0}$
		\State $\var{nodeOrder} \gets \var{message.nodeOrder}$
		\While {$\var{nodeOrder + pow(k, i)} < \var{endID}$}
		\State $\var{currentID = nodeOrder + pow(k, i)}$
			\If{$\var{pow(k, i) <= nodeOrder}$}
			\State $\var{i++}$
			\State $\textbf{continue}$
			\Else
			\State $\var{targetNodeID = NodeID + pow(k, i)}$
			\If{$\var{targetNodeID > endID}$}
			\State $\var{targetNodeID -= endID + 1}$
			\EndIf
			\State $\var{args=(targetNodeID, currentLevel)}$
			\State $\var{receiver} \gets (\var{this.getPeer(args)})$
			\State $\var{i++}$
			\State $\var{sendTo(receiver, message)}$
			\EndIf
		\EndWhile
		\EndFunction
	\end{algorithmic}
\end{algorithm}
\fi

\subsection{Security Consideration} \label{security}

\xxx uses Intel SGX attestations to build trust base among blockchain nodes. The routing information are well saved by Intel SGX enclaves. The behavior of each node after receiving a message or when sending a message are protected by the enclaves. In addition, the parameter k used in within-ring broadcast and the IDs of the contact nodes are randomly generated by using Intel SGX \texttt{sgx\_read\_rand} API.

To hide the existence of \xxx contact nodes from outsiders, there are two mechanisms. First, fake messages are used to make the behavior of a contact node the same with that of a normal node. For a contact node, it will send messages of the same size to both one of the contact nodes and some of the past contact nodes in the upper level ring. For a normal node, it will send messages of the same size to both one of the contact nodes and some of its peers in the same level ring. When broadcasting messages withing the ring, the random parameter k will hide the relative orders of each node. Apart from broadcasting to its children in the constructed k-ary distributed spanning tree, each node will randomly choose a node to send fake message. All fake messages are of the same size of the real messages. Hackers may watch the packets sending out from a node and sending to the node for a long time. However, during one contact node service term, no difference can be discovered from the data collected. Second, contact nodes of a ring will be elected for every service term. One contact node of the last service term will generate random IDs by using Intel SGX's \texttt{sgx\_read\_rand ()} function \cite{costan2016intel}, and then broadcast the nomination result withing the ring. Due to both limited contact node service term period and the same behavior during one service term, outsider cannot differentiate contact nodes from normal nodes. \xxx's network topology and structure are well hidden from the outsiders. Packet analysis is done by using WireShark and the evaluation result is discussed in Section [TODO].

\section{Proof and Analysis}
This is the proof and analysis. \\

\subsection{Proof of Broadcast Performance}

give a proof of time complexity and message comlexity of broadcast, node join and contact node election. Compare to current works.

\subsection{Security and Robustness Analysis}

analyze of fault tolerance, anonymity.

\section{Implementation Details}
We have implemented \xxx in C++ with the only external dependence being \texttt{Protobuf}. The \xxx implementation contains roughly 3800 lines of code, not including tests, configuration on docker, comments and blank lines. An \xxx application running on a server (node) mainly includes three components, a \texttt{NodeTable}, a \texttt{PeerManager}, and a \texttt{Discoverer}. The \texttt{Discoverer} is responsible for bootstrapping the node. The \texttt{NodeTable} stores peer information at each level and maintains the network structure. The \texttt{PeerManager} deal with incoming messages and is responsible for broadcasting messages.

The source code of \xxx will be available at \texttt{github} \texttt{.com/hku-systems/hgfrr} if appropriate.

\section{Evaluation}
This is the evaluation. \\

how we evaluated hgfrr.

\subsection{Evaluation Setup}

talk about how we set up the evaluation

\subsection{Ease of Use}

use kad+gossip to substitute eth

\subsection{Performance Improvements}

talk about the performance improvement in terms of convergence time, message complexity.

\subsubsection{Broadcast}

broadcast performance

\subsubsection{Robustness and Scalability}

fault-tolerance and scalability of broadcast

\subsubsection{Contact Node Election}

performance of contact node election

\subsubsection{Node Join and Leave}

performance of node join and leave, which may lead to structure change

\subsection{Security Evaluation}

We uses WireShark \cite{chappell2010wireshark} to watch the packet sending out from a node and heading to the node. We counts three types of packets: around 17 kb, around 200 bytes, and less than 150 bytes, since they represents three types of messages correspondingly, i.e. the block, the transaction, and other messages including control messages or membership messages. We found that during one service term of a contact node, contact nodes of a ring cannot be differentiated from the normal nodes. By watching and recording for a long time, we found that the overall packet statistics show that all nodes have same percentages of sent and received three types of messages (See Table \ref{tab:packet1} and Table \ref{tab:packet2}). Therefore, due to the short service term of contact nodes and the similar percentages of all types of messages, it is hard for an outsider of the system to differentiate contact nodes from normal nodes.

\begin{table}
	\begin{tabular}{l*{6}{c}r}
		Node Type & $\sim17 KB$ & $\sim200 B$ & $<150 B$ \\
		\hline		
		Normal node in one term & 33.10\%	& 58.60\% & 5.90\% \\
		Contact node in one term & 34.00\%	& 61.20\% &4.50\%  \\
		Node at all time  & 33.70\%	& 60.90\% &4.60\%  \\
	\end{tabular}
	\caption{Send-Packet Analysis of Node in HGFRR}
	\label{tab:packet1}
	\vspace{2mm}
	\begin{tabular}{l*{6}{c}r}
		Node Type & $\sim17 KB$ & $\sim200 B$ & $<150 B$ \\
		\hline		
		Normal node in one term & 35.50\% & 58.60\% & 5.60\% \\
		Contact node in one term & 34.40\% & 59.20\% & 4.70\%  \\
		Node at all time  & 34.60\% & 59.10\% & 5.10\%  \\
	\end{tabular}
	\caption{Receive-Packet Analysis of Node in HGFRR}
	\label{tab:packet2}
\end{table}

\section{Related Work}
This is the related work. \\

More fascinating text. Features\endnote{Remember to use endnotes, not footnotes!} galore, plethora of promises.\\

\section{Conclusion}
This paper identifies the problem of using Gossip in the P2P network layer of a blockchain system, which to some extent already became the bottleneck of further improving the transaction rate. Though tremendous work has been proposed in consensus layer to improve transaction rate, few work addressed the problem from the P2P network layer. HGFRR makes use of the recursive structure to broadcast messages so that messages redundancy could be reduced to the lowest. HGFRR also take the geographical locality of each node into consideration to further improve the efficiency of a broadcast operation. HGFRR is robust even in a dynamic network environment and is hidden from outsiders. Although HGFRR is not as robust as Gossip, it is evaluated and analyzed that the robustness of HGFRR is sufficient in blockchain system context. By trading robustness, HGFRR improve the throughput of blockchain systems by increasing broadcast efficiency. In the future, the transaction rate of the blockchain systems can be improved further in a low bandwidth-consumption or crowded network.

\section*{Acknowledgments}

A polite author always includes acknowledgments.  Thank everyone,
especially those who funded the work. 

%%%%%%%%%%%%%%%%%%%%%% End %%%%%%%%%%%%%%%%%%%%%%%%%

\section*{ATC Template For Reference}

More fascinating text. Features\endnote{Remember to use endnotes, not footnotes!} galore, plethora of promises.\\

%%%%%%%%%%%%%%%%%%%%%% End %%%%%%%%%%%%%%%%%%%%%%%%%

{\normalsize \bibliographystyle{acm}
\bibliography{./main}}

\theendnotes

\end{document}